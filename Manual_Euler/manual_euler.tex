\documentclass[12pt]{article}
\usepackage[brazil]{babel} % para relatórios em português
\usepackage[utf8]{inputenc} % para acentuação direta
\usepackage{amsmath,amsfonts,amssymb}  % improve math presentation
\usepackage{tabularx} % extra features for tabular environment
\usepackage{graphicx} % takes care of graphic including machinery
\usepackage[margin=0.8in,letterpaper]{geometry} % decreases margins
\usepackage[final]{hyperref} % adds hyper links inside the generated pdf file
\usepackage{xcolor}
%\usepackage[pdftex]{hyperref}

\hypersetup{
	colorlinks=true,       % false: boxed links; true: colored links
	linkcolor=blue,        % color of internal links
	citecolor=blue,        % color of links to bibliography
	filecolor=magenta,     % color of file links
	urlcolor=blue         
}

\begin{document}

\title{Manual para usar o HigFlow no cluster Euler}
\author{HigFlow e HigTree}
\date{\today}
\maketitle

\section{Cluster - Passos iniciais}\label{sec:clu_pas_ini}
Considerando que se tenha uma conta no cluster, os passos iniciais são:
\begin{itemize}
	\item \textbf{} $:=$ 	
\end{itemize}

\section{Job e outras funcionalidades}\label{sec:job_fun}

O primeiro passo para se ter acesso ao projeto HigFlow no GitHub é solicitar a permissão ao administrador do projeto, Prof. Dr. Antonio Castelo Filho. Após receber o convite e aceita-lo, siga os passos abaixo pra ter uma versão em sua maquina local:

\begin{itemize}
	\item Crie um diretório onde receberá os arquivos. Neste diretório, em um terminal, digite os comandos abaixo:
	
	\item \textbf{git clone https://github.com/antoniocastelofilho/HigFlow.git}
	
	\item \textbf{git pull origin master}
	
	\item Quando possuir uma versão amplamente testada e estável pode-se fazer 
	
	\item \textbf{git push -u origin master}
	
\end{itemize}

%\bibliographystyle{plain}
%\bibliography{cluster_euler}

\end{document}
